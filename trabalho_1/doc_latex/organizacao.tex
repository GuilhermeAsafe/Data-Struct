 \documentclass[12pt]{article}
 
 % Pacotes
 \usepackage[utf8]{inputenc}
 \usepackage[T1]{fontenc}
 \usepackage[brazil]{babel}
 \usepackage{amsmath}
 \usepackage{graphicx}
 \usepackage{hyperref}
 \usepackage{lipsum} % Para gerar texto de exemplo
 
 % Informações do documento
 \title{Organização TP1}
 
 \begin{document}
 	
 	\maketitle
 	
 	\begin{section}{Implementação}
 		Organizar primeiro os TAD's (já terei decidido qual tipo de ordenação usar) depois pastas e arquivos. Com o básico feito partimos para o \textit{makefile} conseguir compilar o código. Professor falou para implementar listas para testar oque melhor se encaixa, rodar e compilar... base.
    	\end{section}
    
    \begin{section}{Tratamento de entrada (armazenamento)}
    	Devemos escolher como armazenar os dados de entrada que serao as imagens. Lista, pilha, fila? Temos que ordenar o vetor de acordo com a ordem da interceção do elemento $j$ com os $j-1$, sendo $j$ o 'primeiro' elemento do vetor.
    	\end{section}
    	
    \begin{section}{Projeção na reta profundidade}
    	Pense em uma pessoa olhando por cima dessas cenas, onde cada uma diferente e os filamentos são um plastico fino quase transparente. Nesse caso vermelho com azul resultga em roxo. Mas se azul vem antes do vermelho então o resultado da sena final é vermelho, com ajuste do tamanho do intervalo. Seria uma boa ideia para resolver o problema?
    	\end{section}
    
    \begin{section}{Movimento}
    	Entrada com um M para o movimento da cena, ou seja, vamos ter que buscar na lista, pilha, fila o elemento para poder e mudar o intervalo, somando ou subtraindo
    	\end{section}
    	
    	
    \begin{section}{Comparações}
    	O pior a ser implementado é uma algoritimo (até então) $\mathcal{O}(n^2)$, comparando $jn$ com $n-1$ restantes. Mas essa comparação envolve outras opeçaões como cortar o intervalo onde existe interceção.
    \end{section}
    	
    \begin{section}{\texttt{DELETE}}
    	Assuma que $\psi$ seja uma sena que antecede $j$ e o intervalo $[a,b]$ de psi pertence ao intervalo $[a^{'},b^{'}]$ de $j$, nesse caso, devemos implementar \texttt{DELETE $\psi$}. Uma boa escolha do tipo do TAD's é importante, queremos o menor  $\mathcal{O}(f(n))$. 
    	\end{section}
    	
    	
 
 	
 \end{document}
 